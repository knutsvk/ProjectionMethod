\documentclass[final,3p,twocolumn]{elsarticle}

\usepackage{amssymb}
\usepackage{mathtools}
\usepackage{tikz}
\usepackage{pgfplots}
\usepackage{bm}

\biboptions{super, square, sort&compress}

\usepgfplotslibrary{groupplots, external, colormaps}
\tikzexternalize[prefix=Figures/]

\pgfplotscreateplotcyclelist{knuts}
{
    red,mark=x\\%
    blue,mark=o\\%
    teal,mark=triangle\\%
    purple,mark=*,mark size=.5pt\\%
    black,mark=diamond*,mark size=3pt\\%
}

\pgfplotsset
{   
    cycle list name=knuts, 
    width=.9\columnwidth, 
    height=.9\columnwidth, 
    compat=1.12
}

\setcounter{secnumdepth}{3}

\journal{MPhil in Scientific Computing}

\begin{document}

\begin{frontmatter}

\title{Projection Method for Incompressible Flow}

\author{Knut Sverdrup}

\begin{abstract}
    In this report, we discuss computational methods for solving the
    incompressible Navier-Stokes equations in the two-dimensional test case of
    a lid-driven cavity. For Newtonian fluids, we demonstrate the
    implementation of a fractional-step projection method for solving the
    transient problem with moderately high Reynolds numbers, and show that the system
    converges to the steady-state solution previously documented in the
    literature. Additionaly, we discuss the case of Reynolds number zero in
    Bingham plastic fluids, and how the steady-state solution can be computed
    through the SIMPLE algorithm when regularization is employed to deal with
    the singularity in apparent viscosity of such fluids. 
\end{abstract}

\end{frontmatter}

\section{Introduction}
\label{sec:introduction}

In physics, few equations span as wide a range of applications as the
Navier-Stokes equations of fluid dynamics. They govern the motion of viscous
fluids in a continuum framework, and as such have applications in e.g.\ climate
modelling \cite{marshall1997finite, giraldo2008study}, aerodynamics
\cite{rai1987navier, thomas1990navier, jameson1998optimum}, medicinal research
\cite{peskin1977numerical, mihaescu2008large} and petroleum engineering
\cite{deiber1979flow, vinay2006numerical, cardenas2007navier, boyer2010cahn},
to name a few. Named after Claude Navier and George Stokes for their major
contributions \cite{navier1822memoire, stokes1845theories} to its formulation
in the first half of the nineteenth century, the equations have constituted a
major field of research in their own right since their formulation, and
continue to do so today. Apart from their numerous applications, the equations
are fundamentally interesting from a mathematical point of view. In fact,
(dis-)proving the existence and uniqueness of their solutions is one of the
seven Millenium prize problems \cite{fefferman2006existence} for which a prize
of one million US dollars is associated. 

The generalized Navier-Stokes equations are based on the conservation of mass
through the continuity equation,

\begin{equation}
    \partial_t \rho + \nabla \cdot (\rho {\bm u}) = 0 \,,
    \label{eq:continuity}
\end{equation}
%
in addition to conservation of momentum as given by the Cauchy momentum equation

\begin{equation}
    \rho \left( \partial_t \bm{u} + \bm{u} \cdot \nabla \bm{u} \right) 
    = - \nabla p + \nabla \cdot \bm{\tau} + \bm{f} \,.
    \label{eq:cauchy}
\end{equation}
%
Here, we have introduced the primitive variables density $\rho$, velocity
$\bm{u}$ and pressure $p$, in addition to the deviatoric stress tensor
$\bm{\tau}$. The vector $\bm{f}$ accounts for external body sources such as
gravity acting on the fluid, and shall henceforth be disregarded. The
Navier-Stokes equations are an extension of Eqns.\ \eqref{eq:continuity} and
\eqref{eq:cauchy} derived under the assumptions that the stress tensor
$\bm{\tau}$ is a linear function of the strain tensor $\dot{\bm{\gamma}}=\nabla
\bm{u} + (\nabla \bm{u})^T$, that $\nabla \cdot \bm{\tau} = 0$ for fluids at
rest, and that the fluid is isotropic. Given these assumptions, Eq.\
\eqref{eq:cauchy} can be rewritten

\begin{equation}
    \rho \left( \partial_t \bm{u} + \bm{u} \cdot \nabla \bm{u} \right) 
    = - \nabla p + \nabla \cdot \left( \mu \dot{\bm{\gamma}} \right) + \nabla
    \left( \lambda \nabla \cdot \bm{u} \right)
    \label{eq:compressible}
\end{equation}
%
where $\mu$ and $\lambda$ are the first and second coefficients of viscosity,
respectively. Fluids which obey the Navier-Stokes equations as given by Eqns.\
\eqref{eq:continuity} and \eqref{eq:compressible} are labelled Newtonian
fluids.  The second coefficient of viscosity, $\lambda$, is related to bulk
viscosity and disappears for incompressible flow, which we restrict ourselves
to in this project. Consequently, we simply refer to $\mu$ as the
viscosity of the fluid. 

When the density, $\rho$, is constant within each control volume of the fluid,
the flow is said to be isochoric or incompressible. A vast amount of cases in
continuum mechanics relate to incompressible flow, and several simplifications
arise in the description of the fluid. Firstly, Eq.\ \eqref{eq:continuity}
reduces to the incompressibility constraint

\begin{equation}
    \nabla \cdot \bm{u} = 0 \,.
    \label{eq:incompressibility}
\end{equation}
%
Secondly, the viscosity $\mu$ is constant for incompressible flow.
Consequently, Eq.\ \eqref{eq:compressible} simplifies to 

\begin{equation}
    \rho \left( \partial_t \bm{u} + \bm{u} \cdot \nabla \bm{u} \right) 
    = - \nabla p + \mu \nabla^2 \bm{u} \,,
    \label{eq:incompressible}
\end{equation}
%
where we have used the fact that $\nabla \cdot \dot{\bm{\gamma}} = \nabla^2
\bm{u}$ for incompressible flow. Eqns.\ \eqref{eq:incompressibility} and
\eqref{eq:incompressible} make up the incompressible Navier-Stokes equations
for incompressible flow. 

In order to nondimensionalize our system of equations, we move to dimensionless
variables such that $\bm{u} \rightarrow \frac{1}{a} \bm{u}$, $\partial_t \rightarrow
\frac{L}{a} \partial_t$ and $\nabla \rightarrow L \nabla$, where $L$ is
the length of the system under consideration and $a$ is the maximum absolute
value of the velocity. Applying these changes to Eq.\ \eqref{eq:incompressible}
and multiplying through by $L/\rho a^2$ gives the dimensionless equation 

\begin{equation}
    \partial_t \bm{u} + \bm{u} \cdot \nabla \bm{u} = - \frac{1}{\rho a^2}
    \nabla p + \frac{1}{\rm Re} \nabla^2 \bm{u} \,.
    \label{eq:dimensionless}
\end{equation}
%
Here, ${\rm Re} = \rho a L/\mu$ is the Reynolds number of the flow, which
equals the ratio of inertial forces to viscous ones. Note that the pressure
is still measured in physical units, as we wish to utilize different scalings
depending on the Reynolds number. Low Reynolds numbers correspond to laminar
flow, where the viscous forces are dominant. In this case, we let $p
\rightarrow \frac{\rm Re}{\rho a^2} p$, so that Eq.\ \eqref{eq:dimensionless}
becomes

\begin{equation}
    {\rm Re} \left( \partial_t \bm{u} + \bm{u} \cdot \nabla \bm{u} \right)  = -
    \nabla p + \nabla^2 \bm {u} \,.
    \label{eq:loRe}
\end{equation}
%
High Reynolds numbers, on the other hand, correspond to turbulent flow with the
most important contributions arising from the inertial forces. We then let $p
\rightarrow \frac{1}{\rho a^2} p$, yielding the alternative formulation 

\begin{equation}
    \partial_t \bm{u} + \bm{u} \cdot \nabla \bm{u} = - \nabla p
    + \frac{1}{\rm Re} \nabla^2 \bm {u} \,.
    \label{eq:hiRe}
\end{equation}
%
In our implementation of fractional step projection methods for incompresible
flow of Newtonian fluids, Eq.\ \eqref{eq:hiRe} is the formulation of interest.
When discussing creeping flow for Bingham plastic fluids, however, we return to
(a variant of) Eq.\ \eqref{eq:loRe}. 

Several models exist to describe different types of Non-Newtonian fluids, each
characterized by the way the stress $\tau = \sqrt{\frac{1}{2} \sum
\bm{\tau}_{i,j}^2}$ depends on the strain rate $\dot{\gamma} =
\sqrt{\frac{1}{2} \sum \dot{\bm{\gamma}}_{i,j}^2}$. The interesting quantity is
the apparent viscosity $\eta = \tau/ \dot{\gamma}$. For Newtonian fluids, as we
have already seen, the relationship is linear, with slope $\eta = \mu$.
Non-Newtonian fluids include dilatants (shear-thickening, $\partial
\eta/\partial \dot{\gamma} > 0$) and pseudoplastics (shear-thinning, $\partial
\eta/\partial \dot{\gamma} < 0$), but in this project we restrict ourselves to
the relatively simple case of Bingham plastics. Figure \ref{fig:fluids}
illustrates the different types of fluids. 

\begin{figure}[htb]
    \centering
    \tikzsetnextfilename{Fluids}
    \begin{tikzpicture}
        \begin{axis}
            [
                height=0.8\columnwidth,
                axis x line=bottom, 
                xlabel=$\dot{\gamma}$, xmin=0, xmax=1,
                x label style={at={(1.05,0)}},
                axis y line=left,
                ylabel=$\tau$, ymin=0, ymax=2,
                y label style={at={(0,1.05)}},
                ylabel style={rotate=-90},
                ticks=none
            ]
            \addplot+[mark=none, color=black]{x};
            \node[anchor=south, rotate=23] at (axis cs:0.8,0.78) 
                {\footnotesize Newtonian};
            \addplot+[mark=none, color=black]{x+1};
            \node[anchor=south, rotate=23] at (axis cs:0.49,1.46) 
                {\footnotesize Bingham plastic};
            \addplot+[mark=none, color=black, samples=1000]{x+0.5*x^2};
            \node[anchor=south, rotate=40] at (axis cs:0.73,0.97) 
                {\footnotesize Dilatant};
            \addplot+[mark=none, color=black, samples=1000]{x-0.4*x^2};
            \node[anchor=south, rotate=8] at (axis cs:0.83,0.52) 
                {\footnotesize Pseudoplastic};
        \end{axis}
        \node[anchor=south] at (-0.3,2.1) {$B$};
        \draw (-0.1,2.335) -- (0,2.335);
    \end{tikzpicture}
    \caption{Classification of fluids based on apparent viscosity.}
    \label{fig:fluids}
\end{figure}

Bingham plastic fluids, named after Eugene Cook Bingham for his investigation
into them in 1916 \cite{bingham1916investigation}, have a threshold stress
$\tau_0$, below which they do not yield to applied forces. For the
nondimensionalized stress tensor, the yield stress corresponds to the
dimensionless Bingham number $B = \tau_0 L/\mu a$. In other words, the strain
rate is zero unless a stress higher than that characteristic stress is applied.
Physically, this means that they behave as solids for small stresses, something
which leads to interesting behaviour such as non-flat surfaces at rest. This
can be demonstrated by distorting the surface of a mayonnaise container:
gravity alone is not strong enough to surpass the yield stress, and the surface
remains in its distorted state.  Regions where the flow is such that $\tau < B$
are also known as unyielded regions. The relationship between stress and shear
rate for Bingham plastics is therefore

\begin{equation}
    \dot{\gamma} = \begin{cases}
        0 \,,& \tau \leq B \\ 
        \tau - B \,,& \tau > B \end{cases} \,,
    \label{eq:bingham}
\end{equation}
%
from which it is evident that the apparent viscosity 

\begin{equation}
    \eta = \frac{B}{\dot{\gamma}} + 1 \,,
    \label{eq:binghamViscosity}
\end{equation}
%
and as such has a singularity for $\dot{\gamma}=0$. 

As a test framework for our numerical schemes, we apply them to the so-called
lid-driven cavity problem. The problem has served as a benchmark test for
viscous, incompressible fluid flow for decades, and reference solutions are
readily available, notably those produced by Ghia et al.\ in 1982
through the vorticity-stream function formulation and multigrid methods
\cite{ghia1982high}. This test case is fairly simple, and consists of a square
domain in two dimensions with sides of length $L$. At all sides, both solid-wall and
no-slip boundary conditions are applied, so that the component of $\bm{u} =
(u,v)^T$ normal to the wall is zero on the whole boundary, while the tangential
component equals the velocity of wall. In the lid-driven cavity test, all walls
are stationary except the top one (the ``lid''), which moves with constant
speed $a$. Figure \ref{fig:lidDrivenCavity} exhibits the test case
schematically. 

\begin{figure}[htb]
    \centering
    \tikzsetnextfilename{lidDrivenCavity}
    \begin{tikzpicture}
        \draw[thick,black, fill=blue!10!white] (0,0) rectangle (4,4);

        \draw[<-,thick] (3,2) arc (0:330:1);
        \draw[->,thick] (1,4.2) -- (3,4.2);
        \draw[->] (4,0) -- (5,0) node[right] {$x$};
        \draw[->] (0,4) -- (0,5) node[above] {$y$};

        \draw[] (4,0) -- (4,-0.1);
        \draw[] (0,4) -- (-0.1,4);
        \node[anchor=north] at (4,-0.1) {$L$};
        \node[anchor=east] at (-0.1,4) {$L$};

        \draw[] (0,0) -- (0,-0.1);
        \draw[] (0,0) -- (-0.1,0);
        \node[anchor=north] at (0,-0.1) {0};
        \node[anchor=east] at (-0.1,0) {0};

        \node[anchor=north] at (2,-0.1) {$u=0,\,v=0$};
        \node[anchor=south] at (2,4.2) {$u=a,\,v=0$};
        \node[anchor=east] at (-0.1,2.2) {$u=0,$};
        \node[anchor=east] at (-0.2,1.8) {$v=0$};
        \node[anchor=west] at (4.1,2.2) {$u=0,$};
        \node[anchor=west] at (4.12,1.8) {$v=0$};
    \end{tikzpicture}
    \caption{Lid-driven cavity test case.}
    \label{fig:lidDrivenCavity}
\end{figure}

In Section \ref{sec:numerical}, the numerical methods employed to solve the
incompressible Navier-Stokes equations for Newtonian fluids in the lid-driven
cavity are explained. In addition to this, the extensions needed to simulate
creeping flow of Bingham plastics are discussed. Section \ref{sec:results}
contains our results with accompanying comments. Suggestions for possible
extensions are given in Section \ref{sec:further}, while Section
\ref{sec:conclusion} concludes the report. 

\section{Numerical methods}
\label{sec:numerical}

\subsection{Newtonian fluid}
\label{subsec:newtonian}

For the case of a Newtonian fluid, our aim is to find the solution of the
unsteady (transient) Navier-Stokes equations. The coupling of the velocity and
pressure poses a difficulty however, and must be approached in a manner which
allows consistent discretization in time. Subsequently, the spatial domain must
be considered, and the resulting systems of equations solved.  We presently
explain the chosen techniques to do so, and include notes on how they have been
implemented. 

\subsubsection{Projection method}

In order to solve the transient problem given by Eqns.\ \eqref{eq:hiRe} and
\eqref{eq:incompressibility}, we implement a fractional-step projection method
following section 10.3 of Oleg Zikanov's {\em Essential Computational Fluid
Dynamics} \cite{zikanov2010essential}. The original projection method for
Navier-Stokes equations was invented by Chorin in 1968
\cite{chorin1968numerical}, and the essence is still the same in modern
variants. It is an operator splitting approach, in which, for each time step,
one first performs an intermediate time step while ignoring the pressure
forces, and then ignores the viscous forces in the second time step.
Temporal discretization of Eq. \eqref{eq:hiRe} is done explicitly for the
nonlinear term and implicitly for the linear ones, yielding the two equations 

\begin{align}
    \label{eq:firstStep}
    \frac{\tilde{\bm{u}}^{n+1} - \bm{u}^n}{\Delta t} + \bm{u}^n \cdot \nabla
    \bm{u}^n - \frac{1}{\rm Re} \nabla^2 \tilde{\bm{u}}^{n+1} &= 0 \,, \\
    \label{eq:secondStep}
    \frac{\bm{u}^{n+1} - \tilde{\bm{u}}^{n+1}}{\Delta t} + \nabla p^{n+1} &= 0
    \,,
\end{align}
%
which reduce to the discrete version of Eq. \eqref{eq:hiRe} when added
together. Here, we have introduced the notation $\bm{u}^n = \bm{u}(n \Delta
t)$, in addition to denoting by $\tilde{\bm{u}}^{n+1}$ the velocity at the
intermediate time step.  It is particularly important to utilize the implicit
temporal discretization for flows with low Reynolds numbers and for meshes that
are stretched near boundaries, because otherwise the numerical viscous stability
restriction becomes severe in these cases. 

Given a suitable spatial discretization scheme, Eq.\ \eqref{eq:firstStep} is
straightforward to solve since there are two known ($\bm{u}^n$) and two unknown
($\tilde{\bm{u}}^{n+1}$) quantity. The second step, however, has two unknowns.
This problem is solved by taking the divergence of Eq. \eqref{eq:secondStep}
and enforcing $\nabla \cdot \bm{u}^{n+1} = 0$, leading to a Poisson equation
for the pressure:

\begin{equation} 
    \nabla^2 p^{n+1} = \frac{1}{\Delta t} \nabla \cdot \tilde{\bm{u}}^{n+1} \,.
    \label{eq:poissonPressure}
\end{equation}
%
By doing so, the intermediate, unphysical velocity, which was computed without
enforcing incompressibility, is projected onto the space of vector fields
satistfying Eq.\ \eqref{eq:incompressibility}. 

The velocity at time $t^{n+1}$ can finally be updated by rearranging Eq.\
\eqref{eq:secondStep} to read 

\begin{equation}
    \bm{u}^{n+1} = \tilde{\bm{u}}^{n+1} + \Delta t \nabla p^{n+1} \,. 
    \label{eq:velocityUpdate}
\end{equation}

\subsubsection{Finite volumes}

Spatial discretization is done through finite volumes with a staggered grid,
i.e.\ a grid where the three primitive variables are evaluated at
different points.  The need for a staggered grid is due to the relationship
between pressure and velocity. On a regular grid, a checkerboard pattern occurs
in their dependencies upon each other, leading to decoupling and thus spurious
pressure instabilities.  Our domain is split into $N$ equal subdomains for
pressure in each spatial direction, leading to a total of $N^2$ subdomains of
area $\Delta x^2$, where $\Delta x = 1/N$ (we employ the same grid spacing in
both directions). In the centre of each of these squares, the pressure is
evaluated, while the $x$- and $y$-components of velocity are evaluated on their
vertical and horizontal borders, respectively.  Consequently, the control
volumes for the first and second components of the momentum balance equations
in the finite volume method are shifted by $\frac{1}{2} \Delta x$ compared to
the control volumes used for the Poisson equation for pressure. The staggered
grid with shifted control volumes is illustrated in Figure
\ref{fig:staggeredGrid}.

\begin{figure}[htb]
    \centering
    \tikzsetnextfilename{staggeredGrid}
    \begin{tikzpicture}
        \draw[thick,black] (0,0) rectangle (5,5);
        \draw[->] (-0.1,0) -- (5.5,0) node[right] {$x$};
        \draw[->] (0,-0.1) -- (0,5.5) node[above] {$y$};
        \node at (0.5,-0.5) {\footnotesize $i=1$};
        \node at (1.5,-0.5) {\footnotesize $i=2$};
        \node at (3,-0.5) {\footnotesize $\cdots$};
        \node at (4.5,-0.5) {\footnotesize $i=N$};
        \node at (-0.5,0.5) {\footnotesize $j=1$};
        \node at (-0.5,1.5) {\footnotesize $j=2$};
        \node at (-0.5,3) {\footnotesize $\vdots$};
        \node at (-0.5,4.5) {\footnotesize $j=N$};

        \draw[ultra thick,blue] (1,1) rectangle (2,2);
        \draw[ultra thick,dotted,green!50!black] (2.5,1) rectangle (3.5,2);
        \draw[ultra thick,dashed,red] (1,2.5) rectangle (2,3.5);

        \foreach \i in {1,...,4} {
            \draw (\i,0) -- (\i,5);
            \draw (0,\i) -- (5,\i);
        }

        \foreach \i in {0,...,4} {
            \foreach \j in {0,...,4} 
            {
                \node at (\i+0.5,\j+0.5) {\color{blue}{$\circ$}};
            }
        }

        \foreach \i in {0,...,3} {
            \foreach \j in {0,...,4} 
            {
                \node at (\i+1,\j+0.5) {\color{green!50!black}{$\times$}};
                \node at (\j+0.5,\i+1) {\color{red}{$*$}};
            }
        }


    \end{tikzpicture}
    \caption
    {
        Spatial discretization of the square domain with a staggered grid. At the different
        points, the primitive variables are evaluated: \textcolor{blue}{$\circ$}
        $p$, \textcolor{green!50!black}{$\times$} $u$, \textcolor{red}{$*$}
        $v$. Examples of control volumes for pressure ($i=2$, $j=2$),
        $x$-velocity ($i=3.5$, $j=2$) and $y$-velocity ($i=2$, $j=2.5$) are also
        shown with corresponding colors. In this example, $N=5$. 
    }
    \label{fig:staggeredGrid}
\end{figure}

We denote by $\Omega_{i,j}$ the control volume centred at
$(x_i,y_j)=\left(i\Delta x - 1/2, j\Delta x - 1/2\right)$. In
integral form over a control volume centred at a $u$-point, the first component
of Eq.\ \eqref{eq:firstStep} reads 

\begin{multline}
    \int_{\Omega_{i+\frac{1}{2},j}} \tilde{u}^{n+1} {\rm d}V - \frac{\Delta t}{\rm Re}
    \int_{\Omega_{i+\frac{1}{2},j}} \nabla^2 \tilde{u}^{n+1} {\rm d}V \\
    = \int_{\Omega_{i+\frac{1}{2},j}} u^n {\rm d}V + \Delta t
    \int_{\Omega_{i+\frac{1}{2},j}} (u^n \partial_x u^n + v^n \partial_y u^n)
    {\rm d}V \,.
    \label{eq:velocityIntegralForm}
\end{multline}
%
We approximate the first term on each side of the equation by the
two-dimensional midpoint rule, i.e.\ 

\begin{equation}
    \int_{\Omega_{i+\frac{1}{2},j}} \tilde{u}^{n+1} {\rm d}V \approx
    \tilde{u}_{i+\frac{1}{2},j}^{n+1} (\Delta x)^2 \,, 
    \label{eq:firstApprox}
\end{equation}
%
and similar for $u^n$. Derivatives are approximated by central differences, and
when evaluation of a variable is necessary at a point where it is not defined,
the mean value of the nearest neighbours to the point is taken instead. Thus,
the integral in the term with the Laplacian becomes 

\begin{multline}
    \int_{\Omega_{i+\frac{1}{2},j}} \nabla^2 \tilde{u}^{n+1} {\rm d}V =
    \int_{\partial \Omega_{i+\frac{1}{2},j}} \partial_n \tilde{u}^{n+1} 
    {\rm d}S \\
    \approx \tilde{u}_{i+\frac{3}{2},j}^{n+1} +
    \tilde{u}_{i+\frac{1}{2},j-1}^{n+1} + \tilde{u}_{i-\frac{1}{2},j}^{n+1} +
    \tilde{u}_{i+\frac{1}{2},j+1}^{n+1} - 4\tilde{u}_{i+\frac{1}{2},j}^{n+1} \,, 
\end{multline}
%
while the integrals in the nonlinear term are approximated as 

\begin{align}
    \int_{\Omega_{i+\frac{1}{2},j}} u^n \partial_x u^n {\rm d}V \approx
    u_{i+\frac{1}{2},j}^n \frac{1}{2} (u_{i+\frac{3}{2},j}^n -
    u_{i-\frac{1}{2},j}^n) \Delta x \,, \\
    \nonumber 
    \int_{\Omega_{i+\frac{1}{2},j}} v^n \partial_y u^n {\rm d}V \approx
    \frac{1}{2}(u_{i+\frac{1}{2},j+1}^n-u_{i+\frac{1}{2},j-1}^n) \Delta x \\ 
    \label{eq:lastApprox}
    \times \frac{1}{4} (v_{i+1,j+\frac{1}{2}}^n + v_{i+1,j-\frac{1}{2}}^n +
    v_{i,j-\frac{1}{2}}^n + v_{i,j+\frac{1}{2}}^n) \,.
\end{align}
%
By substituting the approximations in Eqns.\ \eqref{eq:firstApprox} -
\eqref{eq:lastApprox} in Eq.\ \eqref{eq:velocityIntegralForm}, we arrive at a
matrix equation of the form 

\begin{equation}
    A_{uv} \tilde{\bf u} = {\bf f}_u + {\bf b}_u \,,
    \label{eq:xVelocityMatrix}
\end{equation}
%
where $A_{uv}$ is an $N(N-1) \times N(N-1)$ matrix, $\tilde{\bf u}$ is a vector
containing the discrete values of $\tilde{u}^{n+1}$ at each point and ${\bf
f}_u$ is a load vector depending only on values of ${\bf u}$ and ${\bf v}$ at
the previous time step. \footnote{Note that bold symbols in italics refer to
physical variables such as $\bm{u}=(u,v)^T$, while roman bold symbols are
vectors of discrete values, e.g.\ ${\bf u} = \{u_{i+\frac{1}{2},j}\}$
for $i \in \{ 1,2,\dots,N-1$ \}, $j \in \{ 1,2,\dots,N \}$.} The vector ${\bf b}_u$ takes into
consideration the boundary conditions, and will be treated shortly. The matrix
is a sparse block tridiagonal matrix of the form

\begin{equation*}
    A_{uv} = 
    \begin{pmatrix}
        I - \alpha B & -\alpha I \\
        -\alpha I & I - \alpha B & -\alpha I \\
        & & \ddots \\
        & & -\alpha I & I - \alpha B
    \end{pmatrix} \,,
\end{equation*}
%
where $\alpha = \frac{\Delta t}{{\rm Re}(\Delta x)^2}$, $I$ is the identity
matrix and

\begin{equation*}
    B = 
    \begin{pmatrix}
        -4 & 1 \\
        1 & -4 & 1 \\
        & & \ddots \\ 
        & & 1 & -4
    \end{pmatrix} \,.
\end{equation*}

By the exact same procedure for the second component of Eq.\
\eqref{eq:firstStep}, using control volumes $\Omega_{i,j+\frac{1}{2}}$, one
arrives at a similar equation for the intermediate $y$-velocities: 

\begin{equation}
    A_{uv} \tilde{\bf v} = {\bf f}_v + {\bf b}_v \,,
    \label{eq:yVelocityMatrix}
\end{equation}

By comparison, the spatial discretization of Eq.\ \eqref{eq:poissonPressure} is
straightforward. In discrete form over a control volume, it reads 

\begin{equation}
    \int_{\Omega_{i,j}} \nabla^2 p^{n+1} {\rm d}V = \frac{1}{\Delta t}
    \int_{\Omega_{i,j}} \nabla \cdot \tilde{\bm{u}}^{n+1} {\rm d}V \,.
\end{equation}
%
We discretize the pressure term as 

\begin{align}
    \nonumber
    \int_{\Omega_{i,j}} & \nabla^2 p^{n+1} {\rm d}V = 
    \int_{\partial \Omega_{i,j}} \nabla p^{n+1} \cdot {\rm d}\bm{S} \\
    \approx& p_{i+1,j}^{n+1} + p_{i,j-1}^{n+1} + p_{i-1,j}^{n+1} +
    p_{i,j+1}^{n+1} -4 p_{i,j}^{n+1} \,,
\end{align}
%
while the term containing the intermediate velocity becomes 

\begin{align}
    \nonumber
    \int_{\Omega_{i,j}} & \nabla \cdot \tilde{\bm{u}}^{n+1} {\rm d}V = 
    \int_{\partial \Omega_{i,j}} \tilde{\bm{u}}^{n+1} \cdot {\rm d}\bm{S} \\
    \approx& (\tilde{u}_{i+\frac{1}{2},j}^{n+1} -
    \tilde{u}_{i-\frac{1}{2},j}^{n+1} + \tilde{v}_{i,j+\frac{1}{2}}^{n+1} -
    \tilde{v}_{i,j-\frac{1}{2}}^{n+1}) \Delta x \,.
\end{align}
%
We thus end up with a system of linear equations: 

\begin{equation}
    A_p {\bf p} = {\bf f}_p + {\bf b}_p \,.
    \label{eq:pressureMatrix}
\end{equation}
%
In this case, there are $N^2$ variables and unknowns, and the matrix is given
by 

\begin{equation*}
    A_p = 
    \begin{pmatrix}
        B & I \\
        I & B & I \\
        & & \ddots \\ 
        & & I & B
    \end{pmatrix} \,.
\end{equation*}

\subsubsection{Boundary conditions}

Special consideration must be taken near the boundaries of the domain. The
update formula for an $x$-momentum control volume requires information taken
from the four nearest $x$-momentum control volumes (north, east, south and
west) in addition to the four nearest $y$-momentum control volumes (northeast,
southeast, southwest and northwest). Upon investigation of Figure
\ref{fig:staggeredGrid}, it is evident that we then need ghost cells for
$\tilde{u}$ at $i=1/2$, $i=N+1/2$, $j=0$ and $j=N+1$. The first two are given
immediately by the Dirichlet conditions (no-slip, solid wall), while the latter
are computed using interpolation. 


\begin{align}
    \nonumber 
    \tilde{u}_{\frac{1}{2},j}^n &= 0 \,, &\quad \tilde{u}_{N + \frac{1}{2},j}^n
    &= 0 \,, \\
    \label{eq:xVelocityBCs}
    \tilde{u}_{i+\frac{1}{2},0}^n &= - \tilde{u}_{i+\frac{1}{2},1}^n \,, &\quad
    \tilde{u}_{i+\frac{1}{2},N+1}^n &= 2a - \tilde{u}_{i+\frac{1}{2},N}^n \,.
\end{align}
%
Utilizing the same technique for $\tilde{v}$, we have 

\begin{align}
    \nonumber
    \tilde{v}_{i,\frac{1}{2}}^n &= 0 \,, &\quad 
    \tilde{v}_{i, N + \frac{1}{2}}^n &= 0 \,, \\
    \label{eq:yVelocityBCs}
    \tilde{v}_{0, j+\frac{1}{2}}^n &= - \tilde{v}_{1, j+\frac{1}{2}}^n \,,
    &\quad \tilde{v}_{N+1, j+\frac{1}{2}}^n &= - \tilde{v}_{N, j+\frac{1}{2}}^n
    \,.
\end{align}

The boundary condiions for $p$ are less obvious, and have been subject to
debate in the literature \cite{liu1995projection}. As there are no physical
boundary conditions on $p$, one could take the inner product of Eq.\
\eqref{eq:hiRe} with either the unit normal or unit tangent at the boundary,
and retrieve equally feasile boundary conditions. In general, the former is
preferred, since the condition relates naturally to the projection operator.
Requiring that the space of divergence-free vector fields is orthogonal to the
space of irrotational vector fields results in the boundary condition
$\bm{u}^{n+1} \cdot \bm{n} = 0$. Taking the inner product of Eq.\
\eqref{eq:secondStep} with $\bm{n}$ thus gives $\partial_n p = 0$ on the domain
border. Our boundary conditions for the pressure are then

\begin{align}
    \nonumber
    p_{0,j}^{n+1} &= p_{1,j}^{n+1} \,, &\quad p_{i,0}^{n+1} &= p_{i,1}^{n+1}
    \,, \quad \\
    \label{eq:pressureBCs}
    p_{N+1,j}^{n+1} &= p_{N,j}^{n+1} \,, &\quad p_{i,N+1}^{n+1} &=
    p_{i,N}^{n+1} \,.
\end{align}

\subsubsection{Eigen for systems of equations}

Each of the linear systems in Eqns.\ \eqref{eq:xVelocityMatrix},
\eqref{eq:yVelocityMatrix} and \eqref{eq:pressureMatrix} are solved using a C++
template library for linear algebra called Eigen \cite{eigenweb}. Since the
matrices are sparse, we exploit the special sparse datastructures available in
Eigen, and solve the systems using a direct simplicial Cholesky factorization
without square root ($A=LDL^T$). The choice of factorization is for efficiency,
and can be used since the sparse matrics are Hermitian (self-adjoint) and
positive definite. 

\subsubsection{Stability and convergence to steady-state}

An important question in any time-marching scheme for partial differential
equations is the restriction on the time step $\Delta t$ necessary to achieve
stability. For a nonlinear scheme such as the one employed, analytical
stability restrictions are not available to the best knowledge of the author,
but it is still possible to gather some insights from investigating the scheme.
The restrictions on $\Delta t$ are dependent on $\alpha = \frac{\Delta t}{{\rm
Re} (\Delta x)^2}$ from the diffusive term and $\beta = \frac{||{\bf
u}||_{\infty} \Delta t}{\Delta x}$ from the convective one. A possibility is
therefore to choose $\Delta t$ as a constant times the minimum of these two
restrictors, i.e.\ 

\begin{equation}
    \Delta t = C \min \left\{ \frac{\Delta x}{||{\bf u}||_{\infty}}, {\rm Re} (\Delta x)^2
    \right\} \,.
    \label{eq:timeStep}
\end{equation}
 
Another approach is to perform stability analysis on a linearized version of
the scheme, and transfer the knowledge to the application of the nonlinear
method. It has been shown that the general form of the dependency of $\Delta t$
on $\Delta x$, $||{\bf u}||_{\infty}$ and ${\rm Re}$ is predicted only somewhat
correctly from this type of procedure \cite{kress2006time}. 

Since no thorough analysis is possible, we have chosen to do a numerical
trial-and-error experiment to find the maximum time step $\Delta t_m$ which
produces stable results for a series of different Reynolds numbers and grid
sizes. We can then choose the time step in a manner which ensures that $\Delta
t < \Delta t_m$. 

If the system converges to steady-state behaviour, the change in the primitive
variables will approach zero. We therefore set a tolerance $\epsilon$ and
advance the system in time until the relative change in the uniform norm of the
vector of discrete velocity values becomes smaller than that tolerance.  In
other words, we require $\frac{||{\bf u}^{n+1} - {\bf u}^{n}||_{\infty}}{
||{\bf u}^{n+1}||_{\infty}} < \epsilon$ for convergence. 

\subsubsection{Vorticity and stream function formulation}

Our reference solutions are those of Ghia et al.\ \cite{ghia1982high},
but their calculations are based on a vorticity and stream function formulation
of the 2D Navier-Stokes equations. In order to compare our results more
easily, we therefore compute these quantities at the end of the
simulation. The scalar stream function $\psi$ is defined for incompressible
flow in 2D through the velocity components such that 

\begin{equation}
    u = \partial_y \psi \,, \quad v = - \partial_x \psi \,.
    \label{eq:streamFunction}
\end{equation}
%
The difference between the stream function at two points equals the
volumetric flux through a line connecting these points. Contour lines of the
stream function are known as stream lines. Stream lines are parallel to the
velocity field at all points in the domain, and trace the movement of particles
in the fluid. Vorticity, on the other hand, describes the local spinning motion
of the fluid at a point. Although a vector quantity in general (the curl of the velocity
field), in 2D there is only one non-zero component. We thus define the
vorticity as 

\begin{equation}
    \omega = |\nabla \times \bm{u}| = \partial_x v - \partial_y u \,,
    \label{eq:vorticity}
\end{equation}
%
which means that computation of the vorticity at grid points
$(x_{i+\frac{1}{2}}, y_{j+\frac{1}{2}})$ in our mesh is straightforward
given the velocity components.  Insertion of Eq.\ \eqref{eq:streamFunction}
into Eq.\ \eqref{eq:vorticity} gives a simple Poisson equation relating $\psi$
to $\omega$: 

\begin{equation}
    \omega = - \nabla^2 \psi
    \label{eq:poissonVorticity}
\end{equation}
%
Conversion from primitive variables to stream function and vorticity can thus
be achieved by first calculating the vorticity at each grid point, and
subsequently solving Eq.\ \eqref{eq:poissonVorticity} with zero boundary
conditions. 

\subsection{Bingham plastic fluid}
\label{subsec:bingham}

Extension of the two-dimensional Navier-Stokes solver to the particular case of
Non-Newtonian fluids known as Bingham plastics introduces two difficulties.
Firstly, the temporal discretization of the viscous term is cumbersome. As introduced in
Section \ref{sec:introduction}, the viscous term $\nabla^2 \bm{u}$ is replaced
by $\nabla \cdot \bm{\tau}$ for these fluids, where 

\begin{equation} 
    \bm{\tau} = \eta(\dot{\gamma}) \bm{\dot{\gamma}} = \left(
    \frac{B}{\dot{\gamma}} + 1 \right) \dot{\bm{\gamma}} \,.
    \label{eq:binghamStress}
\end{equation}
%
It is not possible to discretize this term implicitly in time while retaining
the linearity of the system for $\tilde{\bm{u}}^{n+1}$. Secondly, the
singularity of the apparent viscosity for $\dot{\gamma}=0$ cannot be handled
in a direct manner computationally. 

\subsubsection{Creeping flow}

An idea for circumventing the first of these problems is to discretize the
fluid strain rate tensor implicitly while accepting an explicit temporal
discretization for the strain rate magnitude in the apparent viscosity, in
other words pursuing a discretization such as $\bm{\tau}^{n+\frac{1}{2}} =
\eta(\dot{\gamma}^{n}) \dot{\bm{\gamma}}^{n+1}$. This would result in a
solvable system of equations. As we restrict ourselves to the case of creeping
flows (${\rm Re}=0$), however, this problem vanishes automatically. Equation
\eqref{eq:loRe} applied to the case of Bingham plastic fluids reads 

\begin{equation}
    {\rm Re} \left( \partial_t \bm{u} + \bm{u} \cdot \nabla \bm{u} \right)  = -
    \nabla p + \nabla \cdot \eta ( \dot{\gamma} ) \dot{\bm{\gamma}} \,,
\end{equation}
%
but when ${\rm Re} = 0$, the left side of the equation vanishes and we are left
with the problem of solving 

\begin{equation}
    \nabla p = \nabla \cdot \eta ( \dot{\gamma} ) \dot{\bm{\gamma}} 
    \label{eq:binghamPDE}
\end{equation}
%
under the incompressibility criterion of Eq.\ \eqref{eq:incompressibility}. We
therefore only make a note that computational solutions exist for Bingham
plastics in the transient case (see e.g.\ Vola et al.\
\cite{vola2003laminar} or Muravleva \cite{muravleva2015uzawa}), and concentrate
our efforts on the steady-state case when Eq.\ \eqref{eq:binghamPDE} is
satisfied.

\subsubsection{Regularization}

In order to deal with the singularity in apparent viscosity, a well-tested
technique is regularization. The idea is to replace the discontinuity with a
smooth function which mimics its behaviour. One such regularization technique
which has been used extensively in the literature \cite{mitsoulis2001flow, syrakos2013solution, syrakos2014performance}
is named after Papanastasiou, who authored a paper on it in 1987
\cite{papanastasiou1987flows}. Papanastasiou regularization introduces a
parameter $m$ and replaces the apparent viscosity as given by Eq.\
\eqref{eq:binghamViscosity} by the smooth function 

\begin{equation}
    \eta_p = \frac{B}{\dot{\gamma}} \left(1-e^{-m\dot{\gamma}} \right) + 1
    \label{eq:papa}
\end{equation}
%
which is well-defined also in the limit as $\dot{\gamma}$ approaches zero.
Figure \ref{fig:regularization} exhibits the behaviour of the function for
several values of $m$. Physically, using Eq.\ \eqref{eq:papa} to describe the
apparent viscosity corresponds to replacing the ``solid'' behaviour in the
unyielded regions by a fluid with very high viscosity. When choosing the value
of the parameter $m$, it is obviously desirable to make it large, since this
mimics the physical apparent viscosity better. On the other hand, the exponential in Eq.\
\eqref{eq:papa}, when combined with local numerical errors, causes non-smooth
lines and zig-zagging behaviour for higher $m$ \cite{mitsoulis2001flow}. 

\begin{figure}[htb]
    \centering
    \tikzsetnextfilename{Regularization}
    \begin{tikzpicture}
        \begin{axis}
            [
                width=0.8\columnwidth,
                height=0.6\columnwidth,
                axis x line=bottom, 
                xlabel=$\dot{\gamma}$, xmin=0, xmax=0.8,
                domain=0:0.8,
                x label style={at={(1.05,0)}},
                axis y line=left,
                ylabel=$\tau$, ymin=0, ymax=2,
                y label style={at={(0,1.05)}},
                ylabel style={rotate=-90},
                ticks=none,
                samples=500
            ]
            \addplot[mark=none,color=blue!30!white]{x+1-exp(-4*x)};
            \addplot[mark=none,color=blue!40!white]{x+1-exp(-8*x)};
            \addplot[mark=none,color=blue!50!white]{x+1-exp(-16*x)};
            \addplot[mark=none,color=blue!60!white]{x+1-exp(-32*x)};
            \addplot[mark=none,color=blue!70!white]{x+1-exp(-64*x)};
            \addplot[mark=none,color=blue!80!white]{x+1-exp(-128*x)};
            \addplot[mark=none,color=blue!90!white]{x+1-exp(-256*x)};
            \addplot[mark=none,color=black]{x+1};
        \end{axis}
        \node[anchor=south] at (-0.3,1.3) {$B$};
        \draw (-0.07,1.55) -- (0,1.55) {};
    \end{tikzpicture}
    \caption
    {
        Papanastasiou regularization of the apparent viscosity. The black line
        corresponds to the physical apparent viscosity, whereas blue uses the
        Papanastasiou viscosity $\eta_p$ for various $m$.  Larger values of $m$
        are plotted with darker blue. 
    }
    \label{fig:regularization}
\end{figure}

\subsubsection{Spatial discretization}

Equation \eqref{eq:binghamPDE} is no longer of the form which produces a
checkerboard pattern upon spatial discretization, so there is no longer need
for a staggered grid. We therefore use a standard, colocated grid for $u$, $v$
and $p$ and discretize as before with finite volumes. For each control volume,
Eq.\ \eqref{eq:binghamPDE} in integral form can be written

\begin{equation}
    \int_{\partial \Omega_{i,j}} p {\rm d}\bm{S} = \int_{\partial \Omega_{i,j}}
    \eta \dot{\bm{\gamma}} \cdot {\rm d}\bm{S} \,.
    \label{eq:binghamFVM}
\end{equation}
%
Discretizing the first component of the pressure term in Eq.\
\eqref{eq:binghamFVM} is trivial: 

\begin{equation}
    \int_{\partial \Omega_{i,j}} p {\rm d}\bm{S} \cdot \bm{e}_x \approx
    (p_{i+\frac{1}{2},j}-p_{i-\frac{1}{2},j}) \Delta x \,.
\end{equation}
%
The stress is slightly more cumbersome. We note that, in two dimensions, for
incompressible flow, the strain rate tensor is given by 

\begin{equation}
    \dot{\bm{\gamma}} = 
    \begin{pmatrix}
        \partial_x u - \partial_y v & \partial_y u + \partial_x v \\
        \partial_y u + \partial_x v & \partial_x u - \partial_y v
    \end{pmatrix} \,,
\end{equation}
%
so that its magnitude is 

\begin{equation}
    \dot{\gamma} = \sqrt{(\partial_x u - \partial_y v)^2+(\partial_y u +
    \partial_x v)^2} \,.
    \label{eq:strain}
\end{equation}
%
We approximate the derivatives of the velocities on right part of the boundary $\partial
\Omega_{i,j}$ by 

\begin{align}
    \nonumber 
    \partial_x u_{i+\frac{1}{2},j} &\approx \frac{u_{i+1,j}-u_{i,j}}{\Delta x}
    \,, \\
    \nonumber 
    \partial_y u_{i+\frac{1}{2},j} &\approx
    \frac{u_{i,j+1}+u_{i+1,j+1}-u_{i+1,j-1}-u_{i,j-1}}{4\Delta x} \,, \\
    \nonumber
    \partial_x v_{i+\frac{1}{2},j} &\approx \frac{v_{i+1,j}-v_{i,j}}{\Delta x}
    \,, \\
    \partial_y v_{i+\frac{1}{2},j} &\approx
    \frac{v_{i,j+1}+v_{i+1,j+1}-v_{i+1,j-1}-v_{i,j-1}}{4\Delta x} \,,
\end{align}
%
and similarly for the other edges of the boundary. For brevity, we also denote
by $\dot{\gamma}_{i+\frac{1}{2},j}$ the evaluation of Eq.\ \eqref{eq:strain}
using these approximations. The first component of the viscuous term in Eq.\
\eqref{eq:binghamFVM} can then be approximated as

\begin{align}
    \nonumber 
    \frac{1}{\Delta x} &\int_{\partial \Omega_{i,j}} \eta_p \dot{\bm{\gamma}}
    \cdot {\rm d}\bm{S} \cdot \bm{e}_x \approx \\ 
    \nonumber 
    &+ \eta_p (\dot{\gamma}_{i+\frac{1}{2},j}) (\partial_x
    u_{i+\frac{1}{2},j} - \partial_y v_{i+\frac{1}{2},j}) \\
    \nonumber 
    &- \eta_p (\dot{\gamma}_{i-\frac{1}{2},j}) (\partial_x
    u_{i-\frac{1}{2},j} - \partial_y v_{i-\frac{1}{2},j}) \\
    \nonumber 
    &+ \eta_p (\dot{\gamma}_{i,j+\frac{1}{2}}) (\partial_y
    u_{i,j+\frac{1}{2}} + \partial_x v_{i,j+\frac{1}{2}}) \\
    &- \eta_p (\dot{\gamma}_{i,j-\frac{1}{2}}) (\partial_y
    u_{i,j-\frac{1}{2}} + \partial_x v_{i,j-\frac{1}{2}}) \,.
\end{align}
%
We thus have a discretized version of the $x$-component of Eq.\
\eqref{eq:binghamFVM}, and obtain a similar one by exactly the same procedure
for the $y$-component. The only remaining problem is then to solve the
nonlinear system of discretized equations. 

\subsubsection{Projection methods for steady-state flows}

A popular algorithm for solving nonlinear systems of the form that arises from
our spatial discretization of Eq.\ \eqref{eq:binghamFVM} is the Semi-Implicit
Method for Pressure-Linked Equations (SIMPLE). For a full explanation of the
method and versions of it (SIMPLEC, SIMPLER, \ldots), we recommend Section 10.4
of Zikanov's book \cite{zikanov2010essential}. The basic idea in these methods
is to construct a number of linear systems to be solved iteratively until
convergence is achieved. Linearization of the discretized sets of equations is
done by computing some terms with values from previous iterations.  Successful
implementation of the SIMPLE algorithm to solve the steady-state Navier-Stokes
equations has been demonstrated in the literature \cite{syrakos2013solution}.

\section{Results}
\label{sec:results}

For all results presented in this Section, the velocity of the lid is taken to
be $a=1.0$, while the tolerance required for convergence to steady-state is set
to $\epsilon=10^{-5}$. 

\subsection{Analysis of time step size}
\label{subsec:timeStep}

\begin{figure}[htb]
    \centering
    \tikzsetnextfilename{dt}
    \begin{tikzpicture}
        \begin{axis}
            [
                height=0.8\columnwidth,
                xlabel=Re, xmin=100, xmax=1100, 
                ylabel=$\Delta t_m$, 
                ylabel style={rotate=-90},
                yticklabel style=
                {
                    /pgf/number format/fixed,
                    /pgf/number format/precision=2
                }
            ]
            \addplot+[only marks] table[x={Re}, y={N17}]{../Results/dt.out};
            \addlegendentry{$N=17$};
            \addplot+[only marks] table[x={Re}, y={N33}]{../Results/dt.out};
            \addlegendentry{$N=33$};
            \addplot+[only marks] table[x={Re}, y={N65}]{../Results/dt.out};
            \addlegendentry{$N=65$};
            \addplot[draw=red,dashed,domain=150:1050,samples=100]{500/(17*x)};
            \addplot[draw=blue,dashed,domain=150:1050,samples=100]{500/(33*x)};
            \addplot[draw=teal,dashed,domain=150:1050,samples=100]{500/(65*x)};
        \end{axis}
    \end{tikzpicture}
    \caption
    {
        Maximum time step for convergence to steady-state as a function of the
        Reynolds number. The dashed lines of the same colour as the points
        correspond to our choice of $\Delta t = 500 \Delta x / {\rm Re}$. 
    }
    \label{fig:dt}
\end{figure}

\begin{figure*}[t]
    \centering
    \foreach \t in {1,2,3,4}
    {
        \pgfplotstableread{../Results/psi_N129_Re200_\t.out}{\psi}
        \tikzsetnextfilename{Transient_\t}
        \begin{tikzpicture}
            \begin{axis}
            [
                xlabel={$x$}, ylabel={$y$},
                ylabel style={rotate=-90},
                xmin=0, xmax=1, ymin=0, ymax=1,
                point meta min=-0.12, point meta max=0.02,
                view={0}{90}
            ]
                \addplot3[surf,shader=interp] table[]{\psi}; 
                \addplot3
                [
                    mesh/rows=128, mesh/cols=128, 
                    contour gnuplot=
                    {
                        labels=false, 
                        draw color=black, 
                        levels=
                        {
                            -1e-10, -1e-7, -1e-5, -1e-4, -.01, -.03, -.05, -.07,
                            -.09, -.1, -.11, -.115, -.1175, 1e-8, 1e-7, 1e-6, 1e-5,
                            5e-5, 1e-4, 2.5e-4, 5e-4, 1e-3, 1.5e-3, 3e-3 
                        }
                    }
                ]
                table{\psi}; 
                \addplot[white, quiver={u=\thisrow{u},v=\thisrow{v}, scale arrows=0.5}, -stealth]
                table[each nth point=510]{../Results/vec_N129_Re200_\t.out};
                \addplot[white, quiver={u=\thisrow{u},v=\thisrow{v}, scale arrows=0.5}, -stealth]
                table[each nth point=514]{../Results/vec_N129_Re200_\t.out};
            \end{axis}
        \end{tikzpicture}
    }
    \caption
    {
        Transient results for ${\rm Re}=200$, $N=129$ at different times: $t=0.04$
        (upper left), $t=0.60$ (upper right), $t=1.61$ (lower left) and $t=10$
        (lower right). The surface plots exhibit the stream function $\psi$, its
        contours are stream lines, and the vectors correspond to the velocity
        field $\bm{u}$ at selected points of the domain. 
    }
    \label{fig:transient}
\end{figure*}

Figure \ref{fig:dt} shows the results from our numerical experiment to
determine maximum time step size. As expected, a reduction in temporal step
size is required for increases in both Reynolds number and the amount of grid
points. Based on this trend, we decided to use a formula of the form $\Delta t
\propto \Delta x / {\rm Re}$ when choosing the time step. The dashed lines in
Figure \ref{fig:dt} show corresponding lines with a proportionality factor of
500. With this choice of time step, our simulations remain stable for as large
grid sizes as we have considered ($N$ up to 129) and ${\rm Re} \leq 1000$.
Although we concentrate on the cases where ${\rm Re} \leq 1000$, it is
naturally desirable for the code to work for as high Re as possible. For larger
Reynolds numbers, we therefore employ $\Delta t = 100 \Delta x / {\rm Re}$,
which works (at least) up to $N=129$, ${\rm Re}=3200$. 

\subsection{Transient behaviour}
\label{subsec:transient}

In order to investigate how the impulsively started system evolves to
steady-state, we initiated a test case with ${\rm Re=200}$ and $N=129$,
outputing the system state at even intervals. Figure \ref{fig:transient} shows
the state of the system at four different times through plots of the stream
function and velocity vectors. 

\begin{figure*}[t]
    \centering
    \tikzsetnextfilename{uSlice}
    \begin{tikzpicture}
        \begin{groupplot}
            [
                group style=
                {
                    group size=2 by 2, 
                    xlabels at=edge bottom, 
                    xticklabels at=edge bottom, 
                    ylabels at=edge left, 
                    vertical sep=25pt, 
                    horizontal sep=25pt
                },
                xlabel=$y$, xmin=0, xmax=1, 
                ylabel=$u$, 
                ylabel style={rotate=-90}
            ]
            \nextgroupplot[title={Re=100}]
            \foreach \N in {17,33,65,129}
            {
                \addplot+[only marks] table{../Results/uGC_N\N_Re100.out}; 
            }
            \addplot+[only marks] table[x={y}, y={Re100}]{../Results/uGC.ref};

            \nextgroupplot[title={Re=400}]
            \foreach \N in {17,33,65,129}
            {
                \addplot+[only marks] table{../Results/uGC_N\N_Re400.out}; 
            }
            \addplot+[only marks] table[x={y}, y={Re400}]{../Results/uGC.ref};

            \nextgroupplot[title={Re=1000}]
            \foreach \N in {17,33,65,129}
            {
                \addplot+[only marks] table{../Results/uGC_N\N_Re1000.out}; 
            }
            \addplot+[only marks] table[x={y}, y={Re1000}]{../Results/uGC.ref};

            \nextgroupplot[title={Re=3200}]
            \foreach \N in {17,33,65,129}
            {
                \addplot+[only marks] table{../Results/uGC_N\N_Re3200.out}; 
            }
            \addplot+[only marks, unbounded coords=jump, 
            y filter/.expression={y<-0.5 ? nan : y}] 
            table[x={y}, y={Re3200}]{../Results/uGC.ref};
        \end{groupplot}
    \end{tikzpicture}
    \caption
    {
        First component $u$ of velocity through the slice $x=0.5$. The filled
        black diamonds corresponds to tabulated results from Ghia et al.\
        \cite{ghia1982high}. \textcolor{red}{$\times$} $N=17$,
        \textcolor{blue}{$\circ$} $N=33$, \textcolor{teal}{$\triangle$} $N=65$,
        \textcolor{purple}{\huge $_{\cdot}$} $N=129$.  
    }
    \label{fig:uGC}
\end{figure*}

The system behaves as one would expect. Initially, the fluid is at rest in all
parts of the domain. In the upper left plot of Figure \ref{fig:transient}, the
system has only just started to be affected by the movement of the lid: there
is a flow along the lid and a slight circulation from the upper right corner,
through the center, and back to the upper left corner. In the next snapshot
(upper right), this circulation has become more pronounced, as is evident from
the relative size of the velocity vectors. In the lower left plot, the
formation of a vortex is evident in the upper right corner. Finally, at
steady-state (lower right), the system has evolved to its final configuration,
where a constant circulation is present in the cavity, and we can even discern
the creation of a secondary eddy in the lower right corner. Note that all the
velocity vectors are parallel to the stream lines, as expected. 

\subsection{Steady-state solution}
\label{subsec:steady}

Known reference solutions are also necessary to validate the code. As
such, we compare our findings with those of Ghia et al.\ \cite{ghia1982high},
since their paper is widely recognized as the reference for high-Re
incompressible flow in the lid-driven cavity test case. Results have been
computed with Reynolds numbers of 100, 400, 1000 and 3200 and grid sizes
increasing from $N=17$ to $N=129$. In Figures \ref{fig:uGC} -
\ref{fig:vorticity}, results are presented which visualize the steady-state
distribution of velocities, stream function and vorticity. 

\begin{figure*}[t]
    \centering
    \tikzsetnextfilename{vSlice}
    \begin{tikzpicture}
        \begin{groupplot}
            [
                group style=
                {
                    group size=2 by 2, 
                    xlabels at=edge bottom, 
                    xticklabels at=edge bottom, 
                    ylabels at=edge left, 
                    vertical sep=25pt, 
                    horizontal sep=25pt
                },
                xlabel=$x$, xmin=0, xmax=1, 
                ylabel=$v$, 
                ylabel style={rotate=-90}
            ]
            \nextgroupplot[title={Re=100}]
            \foreach \N in {17,33,65,129}
            {
                \addplot+[only marks] table{../Results/vGC_N\N_Re100.out}; 
            }
            \addplot+[only marks] table[x={x}, y={Re100}]{../Results/vGC.ref};

            \nextgroupplot[title={Re=400}]
            \foreach \N in {17,33,65,129}
            {
                \addplot+[only marks] table{../Results/vGC_N\N_Re400.out}; 
            }
            \addplot+[only marks] table[x={x}, y={Re400}]{../Results/vGC.ref};

            \nextgroupplot[title={Re=1000}]
            \foreach \N in {17,33,65,129}
            {
                \addplot+[only marks] table{../Results/vGC_N\N_Re1000.out}; 
            }
            \addplot+[only marks] table[x={x}, y={Re1000}]{../Results/vGC.ref};

            \nextgroupplot[title={Re=3200}]
            \foreach \N in {17,33,65,129}
            {
                \addplot+[only marks] table{../Results/vGC_N\N_Re3200.out}; 
            }
            \addplot+[only marks] table[x={x}, y={Re3200}]{../Results/vGC.ref};
        \end{groupplot}
    \end{tikzpicture}
    \caption
    {
        Second component $v$ of velocity through the slice $y=0.5$. The filled black
        diamonds corresponds to tabulated results from Ghia et al.\
        \cite{ghia1982high}. \textcolor{red}{$\times$} $N=17$,
        \textcolor{blue}{$\circ$} $N=33$, \textcolor{teal}{$\triangle$} $N=65$,
        \textcolor{purple}{\huge $_{\cdot}$} $N=129$. 
    }
    \label{fig:vGC}
\end{figure*}

Figure \ref{fig:uGC} shows how the first component of the velocity, $u$, varies
in the plane $x=0.5$, a slice through the geometric centre of the computational
domain. Note especially the filled, black diamonds, which correspond to values
tabulated by Ghia et al.\ in their paper \cite{ghia1982high}. For ${\rm
Re} = 100$, our results are very close to the reference even for the smallest
grid sizes. Our solver struggles more as Re increases, but it is still evident
that grid refinement gives convergence. For the highest grid resolution, our
results align perfecly with the reference for ${\rm Re} \leq 1000$, and they
even approach the reference solution for ${\rm Re} = 3200$. Similar results for
the second velocity component, $v$, are given in Figure \ref{fig:vGC}. Here,
the slice is through $y=0.5$. We observe that the results tend towards the
reference solutions in the same manner as for $u$. 

\begin{figure*}[t]
    \centering
    \foreach \Re in {100,400,1000,3200}
    {
        \pgfplotstableread{../Results/psi_N129_Re\Re.out}{\psi}
        \tikzsetnextfilename{Stream_Re\Re}
        \begin{tikzpicture}
            \begin{axis}
            [
                title={${\rm Re}=\Re$},
                xlabel={$x$}, ylabel={$y$},
                ylabel style={rotate=-90},
                xmin=0, xmax=1, ymin=0, ymax=1,
                point meta min=-0.12, point meta max=0.02,
                view={0}{90}
            ]
                \addplot3[surf,shader=interp] table[]{\psi}; 
                \addplot3
                [
                    mesh/rows=128, mesh/cols=128, 
                    contour gnuplot=
                    {
                        labels=false, 
                        draw color=black, 
                        levels=
                        {
                            -1e-10, -1e-7, -1e-5, -1e-4, -.01, -.03, -.05, -.07,
                            -.09, -.1, -.11, -.115, -.1175, 1e-8, 1e-7, 1e-6, 1e-5,
                            5e-5, 1e-4, 2.5e-4, 5e-4, 1e-3, 1.5e-3, 3e-3 
                        }
                    }
                ]
                table{\psi}; 
                \addplot[white, quiver={u=\thisrow{u},v=\thisrow{v}, scale arrows=0.5}, -stealth]
                table[each nth point=510]{../Results/vec_N129_Re\Re.out};
                \addplot[white, quiver={u=\thisrow{u},v=\thisrow{v}, scale arrows=0.5}, -stealth]
                table[each nth point=514]{../Results/vec_N129_Re\Re.out};
            \end{axis}
        \end{tikzpicture}
    }
    \caption
    {
        Surface plot of the stream function $\psi$ for several different
        Reynold's numbers. The stream lines are the contours of $\psi$ used in
        the reference by Ghia et al.\ \cite{ghia1982high} Also plotted are
        selected vectors from the computed velocity field. 
    }
    \label{fig:stream}
\end{figure*}

\begin{figure*}[t]
    \centering
    \foreach \Re in {100,400,1000,3200}
    {
        \pgfplotstableread{../Results/omega_N129_Re\Re.out}{\omega}
        \tikzsetnextfilename{Vorticity_Re\Re}
        \begin{tikzpicture}
            \begin{axis}
            [
                title={${\rm Re}=\Re$},
                xlabel={$x$}, ylabel={$y$},
                ylabel style={rotate=-90},
                xmin=0, xmax=1, ymin=0, ymax=1,
                point meta min=-6, point meta max=6,
                view={0}{90}
            ]
                \addplot3[surf,shader=interp] table[]{\omega}; 
                \addplot3
                [
                    mesh/rows=128, mesh/cols=128, 
                    contour gnuplot=
                    {
                        labels=false, 
                        draw color=black, 
                        levels={-3,-2,-1,-0.5,0,0.5,1,2,3,4,5}
                    }
                ]
                table[]{\omega}; 
            \end{axis}
        \end{tikzpicture}
    }
    \caption
    {
        Surface plot of the vorticity $\omega$ for several different Reynold's
        numbers. The contour lines are the same as those used in the reference
        by Ghia et al.\ \cite{ghia1982high}
    }
    \label{fig:vorticity}
\end{figure*}

Figure \ref{fig:stream} shows the distribution of the stream function and
velocity field at steady-state. When compared to Figure 3 in the paper by Ghia
et al.\, we see that the streamline pattern is near identical for all
cases. The stream lines which we have plotted as contours of $\psi$ are the
same as those used in the reference. Our inclusion of the velocity field at
selected places in the domain is merely to give an intuitive understanding of
the distribution of flow. As the Reynolds number gets larger, we observe that
the secondary eddies become more prominent, in agreement with our expectations
from the literature. Figure \ref{fig:vorticity} depicts the vorticity
distribution in the domain, including contours. Our results are
indistinguishable from the results as shown in Figure 4 in the reference
\cite{ghia1982high}. 

\subsection{Computational efficiency}
\label{subsec:efficiency}

Since computational efficiency is an important factor in our ability to
simulate realistic scenarios, it is beneficial to evaluate the performance of
the code. For each pair of parameters Re and $N$, Table \ref{tab:runtime}
contains the total runtime of the code in seconds, the number of iterations required to
satistfy the steady-state tolerance and the runtime per iteration. The latter
is computed by timing the while-loop only and dividing the result by the
iteration count. 

As is evident, the runtime quickly blows up as the Reynolds number and amount
of grid points both increase. For the highest combination run, the simulation
took just over two hours in total. Both parameters are inversely proportional to our
choice of time step, so more iterations are needed to reach the same time.
Additionally, the total time required to reach steady-state increases with Re.
There is not, however, very much to be done computationally about these
problems, except perhaps finding a choice of time step which is more optimally
chosen as the largest permissible. The average time per iteration, on the other
hand, is determined solely by the choice of spatial accuracy. Comparing the
average runtime per iteration for different values of $N$, we note that it is
approximately of the order $\mathcal{O}(N^2)$, i.e.\ linear in terms of the
matrix sizes for the systems being solved. This is as good as we can hope for
for the problem of solving a system of equations given a decomposed matrix, and
only possible since the matrices are sparse block-tridiagonal in nature. 

\begin{table}[htb]
    \centering
    \begin{tabular}{ccccc}
        \hline
        Re   & $N$ & Runtime & Iter. & Time per iter.          \\
        \hline
        100  & 17  & 0.09    & 58    & 1.20690 $\cdot 10^{-3}$ \\
        400  & 17  & 0.61    & 552   & 1.08696 $\cdot 10^{-3}$ \\
        1000 & 17  & 1.93    & 1710  & 1.12881 $\cdot 10^{-3}$ \\
        3200 & 17  & 29.9    & 26282 & 1.13728 $\cdot 10^{-3}$ \\
        \hline
        100  & 33  & 0.67    & 99    & 4.74747 $\cdot 10^{-3}$ \\ 
        400  & 33  & 4.17    & 841   & 4.73246 $\cdot 10^{-3}$ \\ 
        1000 & 33  & 14.62   & 3035  & 4.75124 $\cdot 10^{-3}$ \\ 
        3200 & 33  & 154.02  & 32644 & 4.71235 $\cdot 10^{-3}$ \\ 
        \hline
        100  & 65  & 6.05    & 174   & 1.97126 $\cdot 10^{-2}$ \\ 
        400  & 65  & 31.3    & 1385  & 2.06426 $\cdot 10^{-2}$ \\ 
        1000 & 65  & 88.99   & 4248  & 2.01154 $\cdot 10^{-2}$ \\ 
        3200 & 65  & 1226.28 & 60264 & 2.03035 $\cdot 10^{-2}$ \\ 
        \hline
        100  & 129 & 68.45   & 308   & 8.82792 $\cdot 10^{-2}$ \\ 
        400  & 129 & 260.37  & 2412  & 9.06758 $\cdot 10^{-2}$ \\ 
        1000 & 129 & 680.18  & 7118  & 8.97275 $\cdot 10^{-2}$ \\ 
        3200 & 129 & 7289.88 & 81448 & 8.90100 $\cdot 10^{-2}$ \\ 
        \hline
    \end{tabular}
    \caption
    {
        Total runtime, number of iterations required to reach steady-state and
        runtime per iteration for the various combinations of Re and $N$. The
        runtimes are measured in seconds. 
    }
    \label{tab:runtime}
\end{table}

\section{Suggestions for further work}
\label{sec:further}

The most obvious next step is to implement the SIMPLE algorithm for solving the
steady-state Navier-Stokes equations for Bingham plastics. This implementation
is not necessarily trivial, especially if trying to do so in a computationally
optimal manner, taking advantage of preconditioners and utilizing multigrid
methods. It would also be very interesting to try and develop a transient
solver of the same type as we have implemented for the Newtonian case,
following the idea of semi-implicit temporal discretization of the viscuous
term. 

For high Reynolds numbers, use of the time-marching PDE solver becomes a very
inefficient method to obtain steady-state solutions. Higher grid sizes and
smaller time steps are necessary for convergence, so obtaining results becomes
impractical rapidly as Re is increased. For the steady-state case, one could
use an algorithm such as SIMPLE or a vorticity-stream-function method.  There
are, however, actions that could be taken to speed up the solver already
implemented. Since each time step in the time-marching scheme is dependent on
the previous one, optimization should be done within each time step. The
matrices for the linear systems are only built once at the beginning of the
simulation, and although the assembling of these could be looked into, the gain
would probably be negligible compared to that achieved by speeding up the
functions used within each time step. 

We have not utilized proper vectorization when updating the load vectors for
the linear systems, but instead calculated each component individually. This
could perhaps reduce the time spent in each time step marginally, but the main
computational effort is in solving the linear systems of equations. It would be
difficult to improve on the actual solvers provided by the Eigen library
(which, as we have seen, provide near-linear solving), but inspection of the
(block) tridiagonal matrices make it apparent that they could be suitable for
variants of the Hockney algorithm \cite{hockney1965fast}, which decouples the
system of equations to $N$ diagonal systems, each easily solved rapidly (in
$\mathcal{O}(N)$ operations). The runtime would then be dominated by $N$ matrix
multiplications, each of the order $\mathcal{O}(N^2)$.  By utilizing the fast
Fourier transform \cite{cooley1965algorithm}, the computational complexity can
be reduced to $\mathcal{O}(N \log N)$ for each of these. Even further
improvement could be possible by noting that the decoupled systems of equations
enable parallelization, meaning that in theory, with optimal paralellization,
the total complexity could be reduced down towards $\mathcal{O}(N \log N)$. 

Finally, steps should be taken to increase the amount of interesting
systems we can look at. Inclusion of arbitrarily shaped domains can be done by
using adaptive mesh refinement \cite{schoch2013eulerian, lovett2015adaptive}
coupled with e.g.\ level set and ghost fluid methods \cite{schoch2013eulerian}
or Cartesian cut cells \cite{klein2009well}, which would also allow for moving
boundaries.  Additionally, extensions to three dimensions would be natural. 

\section{Conclusions}
\label{sec:conclusion}

We have successfully implemented a numerical solver for Newtonian fluids
obeying the unsteady, incompressible Navier-Stokes equations in two spatial
dimensions. Our results for the benchmark problem of a lid-driven cavity are in
agreement with those previously presented in the literature for Reynolds
numbers at least up to 3200, although computations converge slower and become
more expensive quickly as Re increases.  In addition to this implementation, we
have discussed possible extensions to the special class of Non-Newtonian
fluilds known as Bingham plastics in the case of Reynolds number zero (creeping
flow). There is nothing novel about the methods employed and the resulting
code, but this project report nevertheless serves as a literature review which
covers an introduction to the problem at hand, and which validates the
robustness of our code through well-documented tests. 

\section*{References}
\bibliographystyle{elsarticle-num}
\bibliography{references.bib}

\end{document}
