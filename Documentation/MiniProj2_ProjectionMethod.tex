\documentclass[final,3p,twocolumn]{elsarticle}

\usepackage{amssymb}
\usepackage{mathtools}
\usepackage{tikz}
\usepackage{pgfplots}
\usepackage{bm}

\biboptions{super, square, sort&compress}

\usepgfplotslibrary{groupplots, external, colormaps}
\tikzexternalize[prefix=Figures/]

\pgfplotscreateplotcyclelist{knuts}
{
    red,mark=x\\%
    blue,mark=o\\%
    teal,mark=triangle\\%
    purple,mark=*,mark size=.5pt\\%
    black,mark=diamond*,mark size=3pt\\%
}

\pgfplotsset
{   
    cycle list name=knuts, 
    colormap name={viridis},
    width=.9\columnwidth, 
    height=.9\columnwidth, 
    compat=1.12
}

\setcounter{secnumdepth}{2}

\journal{MPhil in Scientific Computing}

\begin{document}

\begin{frontmatter}

\title{Projection Method for Incompressible Flow}

\author{Knut Sverdrup}

\begin{abstract}
\end{abstract}

\end{frontmatter}

\section{Introduction}
\label{sec:introduction}

In physics, few equations span as wide a range of applications as the
Navier-Stokes equations of fluid dynamics. They govern the motion of viscuous
fluids in a continuum framework, and as such have applications in {\em e.g.\ }
climate modelling \cite{marshall1997finite, giraldo2008study}, aerodynamics
\cite{rai1987navier, thomas1990navier, jameson1998optimum}, medicinal research
\cite{peskin1977numerical, mihaescu2008large} and petroleum engineering
\cite{deiber1979flow, vinay2006numerical, cardenas2007navier, boyer2010cahn},
to name a few. Named after Claude Navier and George Stokes for their major
contributions \cite{navier1822memoire, stokes1845theories} to its formulation
in the first half of the nineteenth century, the equations have constituted a
major field of research in their own right since their formulation, and
continue to do so today. Apart from their numerous applications, the equations
are fundamentally interesting from a mathematical point of view. In fact,
(dis-)proving the existence and uniqueness of their solutions is one of the
seven Millenium prize problems \cite{fefferman2006existence} for which a prize
of one million US dollars is associated. 

The generalized Navier-Stokes equations are based on the conservation of mass
through the continuity equation,

\begin{equation}
    \partial_t \rho + \nabla \cdot (\rho {\bm u}) = 0 \,,
    \label{eq:continuity}
\end{equation}
%
in addition to conservation of momentum as given by the Cauchy momentum equation

\begin{equation}
    \rho \left( \partial_t \bm{u} + \bm{u} \cdot \nabla \bm{u} \right) 
    = - \nabla p + \nabla \cdot \bm{\tau} + \bm{f} \,.
    \label{eq:cauchy}
\end{equation}
%
Here, we have introduced the primitive variables density $\rho$, velocity
$\bm{u}$ and pressure $p$, in addition to the deviatoric stress tensor
$\bm{\tau}$. The vector $\bm{f}$ accounts for external body sources such as
gravity acting on the fluid, and shall henceforth be disregarded. The
Navier-Stokes equations are an extension of Eqns.\ \eqref{eq:continuity} and
\eqref{eq:cauchy} derived under the assumptions that the stress tensor
$\bm{\tau}$ is a linear function of the strain tensor $\dot{\bm{\gamma}}=\nabla
\bm{u} + (\nabla \bm{u})^T$, that $\nabla \cdot \bm{\tau} = 0$ for fluids at
rest, and that the fluid is isotropoc. Given these assumptions, Eq.\
\eqref{eq:cauchy} can be rewritten

\begin{equation}
    \rho \left( \partial_t \bm{u} + \bm{u} \cdot \nabla \bm{u} \right) 
    = - \nabla p + \nabla \cdot \left( \mu \dot{\bm{\gamma}} \right) + \nabla
    \left( \lambda \nabla \cdot \bm{u} \right)
    \label{eq:compressible}
\end{equation}
%
where $\mu$ and $\lambda$ are the first and second coefficients of viscosity,
respectively. Fluids which obey the Navier-Stokes equations as given by Eqns.\
\eqref{eq:continuity} and \eqref{eq:compressible} are labelled Newtonian
fluids.  The second coefficient of viscosity, $\lambda$, is related to bulk
viscosity and disappears for incompressible flow, which we restrict ourselves
to in this project. Consequently, we shall simply refer to $\mu$ as the
viscosity of the fluid in the following. 

When the density $\rho$ is constant within each control volume of the fluid,
the flow is said to be isochoric or incompressible. A vast amount of cases in
continuum mechanics relate to incompressible flow, and several simplifications
arise in the description of the fluid. Firstly, Eq.\ \eqref{eq:continuity}
reduces to the incompressibility constraint

\begin{equation}
    \nabla \cdot \bm{u} = 0 \,.
    \label{eq:incompressibility}
\end{equation}
%
Secondly, the viscosity $\mu$ is constant for incompressible flow.
Consequently, Eq.\ \eqref{eq:compressible} simplifies to 

\begin{equation}
    \rho \left( \partial_t \bm{u} + \bm{u} \cdot \nabla \bm{u} \right) 
    = - \nabla p + \mu \nabla^2 \bm{u} \,,
    \label{eq:incompressible}
\end{equation}
%
where we have used the fact that $\nabla \cdot \dot{\bm{\gamma}} = \nabla^2
\bm{u}$ for incompressible flow. Eqns.\ \eqref{eq:incompressibility} and
\eqref{eq:incompressible} make up the incompressible Navier-Stokes equations
for incompressible flow. 

In order to nondimensionalize our system of equations, we move to dimensionless
variables such that $\bm{u} \rightarrow \frac{1}{a} \bm{u}$, $\partial_t \rightarrow
\frac{L}{a} \partial_t$ and $\nabla \rightarrow L \nabla$, where $L$ is
the length of the system under consideration and $a$ is the maximum absolute
value of the velocity. Applying these changes to Eq.\ \eqref{eq:incompressible}
and multiplying through by $L/(\rho a^2)$ gives the dimensionless equation 

\begin{equation}
    \partial_t \bm{u} + \bm{u} \cdot \nabla \bm{u} = - \frac{1}{\rho a^2}
    \nabla p + \frac{1}{\rm Re} \nabla^2 \bm{u} \,.
    \label{eq:dimensionless}
\end{equation}
%
Here, ${\rm Re} = \rho a L / \mu$ is the Reynolds number of the flow, which
equals the ratio of intertial forces to viscuous ones. Note that the pressure
is still measured in physical units, as we wish to utilize different scalings
depending on the Reynolds number. Low Reynolds numbers correspond to laminar
flow, where the viscuous forces are dominant. In this case, we let $p
\rightarrow \frac{\rm Re}{\rho a^2} p$, so that Eq.\ \eqref{eq:dimensionless}
becomes

\begin{equation}
    {\rm Re} \left( \partial_t \bm{u} + \bm{u} \cdot \nabla \bm{u} \right)  = -
    \nabla p + \nabla^2 \bm {u} \,.
    \label{eq:loRe}
\end{equation}
%
High Reynolds numbers, on the other hand, correspond to turbulent flow with the
most important contributions arising from the intertial forces. We then let $p
\rightarrow \frac{1}{\rho a^2} p$, yielding the alternative formulation 

\begin{equation}
    \partial_t \bm{u} + \bm{u} \cdot \nabla \bm{u} = - \nabla p
    + \frac{1}{\rm Re} \nabla^2 \bm {u} \,.
    \label{eq:hiRe}
\end{equation}
%
In our implementation of fractional step projection methods for incompresible
flow of Newtonian fluids, Eq.\ \eqref{eq:hiRe} is the formulation of interest.
When discussing creeping flow for Bingham plastic fluids, however, we return to
(a variant of) Eq.\ \eqref{eq:loRe}. 

Several models exist to describe different types of Non-Newtonian fluids, each
characterized by the way the stress $\tau = \sqrt{\frac{1}{2} \sum
\bm{\tau}_{i,j}^2}$ depends on the strain rate $\dot{\gamma} =
\sqrt{\frac{1}{2} \sum \dot{\bm{\gamma}}_{i,j}^2}$. The interesting variable is
the apparent viscosity $\eta = \tau/ \dot{\gamma}$. For Newtonian fluids, as we
have already seen, the relationship is linear, with slope $\eta = \mu$.
Non-Newtonian fluids include dilatants (shear-thickening, $\partial
\eta/\partial \dot{\gamma} > 0$) and pseudoplastics (shear-thinning, $\partial
\eta/\partial \dot{\gamma} < 0$), but in this project we restrict ourselves to
the relatively simple case of Bingham plastics. Figure \ref{fig:fluids}
illustrates the different types of fluids. 

\begin{figure}[htb]
    \centering
    \tikzsetnextfilename{Fluids}
    \begin{tikzpicture}
        \begin{axis}
            [
                height=0.8\columnwidth,
                axis x line=bottom, 
                xlabel=$\dot{\gamma}$, xmin=0, xmax=1,
                x label style={at={(1.05,0)}},
                axis y line=left,
                ylabel=$\tau$, ymin=0, ymax=2,
                y label style={at={(0,1.05)}},
                ylabel style={rotate=-90},
                ticks=none
            ]
            \addplot+[mark=none, color=black]{x};
            \node[anchor=south, rotate=23] at (axis cs:0.8,0.78) 
                {\footnotesize Newtonian};
            \addplot+[mark=none, color=black]{x+1};
            \node[anchor=south, rotate=23] at (axis cs:0.49,1.46) 
                {\footnotesize Bingham plastic};
            \addplot+[mark=none, color=black, samples=1000]{x+0.5*x^2};
            \node[anchor=south, rotate=40] at (axis cs:0.73,0.97) 
                {\footnotesize Dilatant};
            \addplot+[mark=none, color=black, samples=1000]{x-0.4*x^2};
            \node[anchor=south, rotate=8] at (axis cs:0.83,0.52) 
                {\footnotesize Pseudoplastic};
        \end{axis}
        \node[anchor=south] at (-0.3,2.1) {$B$};
    \end{tikzpicture}
    \caption{Classification of fluids based on apparent viscosity.}
    \label{fig:fluids}
\end{figure}

Bingham plastic fluids, named after Eugene Cook Bingham for his investigation
into them in 1916 \cite{bingham1916investigation}, have a threshold stress
$\tau_0$, below which they do not yield to applied forces. For the
nondimensionalized stress tensor, the yield stress corresponds to the
dimensionless Bingham number $B = \tau_0 L/(\mu a)$. In other words, the strain
rate is zero unless a stress higher than that characteristic stress is applied.
Physically, this means that they behave as solids for small stresses, something
which leads to interesting behaviour such as non-flat surfaces at rest. Regions
where the flow is such that $\tau < B$ are also known as unyielded
regions. The relationship between stress and shear rate for Bingham plastics is
therefore

\begin{equation}
    \dot{\gamma} = \begin{cases}
        0 \,,& \tau \leq B \\ 
        \tau - B \,,& \tau > B \end{cases} \,,
    \label{eq:bingham}
\end{equation}
%
from which it is evident that the apparent viscosity 

\begin{equation}
    \eta = \frac{B}{\dot{\gamma}} + 1 \,,
    \label{eq:binghamViscosity}
\end{equation}
%
and as such has a singularity for $\dot{\gamma}=0$. 

\begin{figure}[htb]
    \centering
    \tikzsetnextfilename{lidDrivenCavity}
    \begin{tikzpicture}
        
    \end{tikzpicture}
    \caption{Lid-driven cavity test case.}
    \label{fig:lidDrivenCavity}
\end{figure}

\begin{itemize}
    \item Introduce the lid-driven cavity test problem, explain what has been
        done in the literature. 
\end{itemize}

\section{Numerical methods}
\label{sec:numerical}

\subsection{Newtonian fluid}
\label{subsec:newtonian}

\begin{itemize}
    \item (Fractional step) projection methods 
    \item Spatial discretization: the finite volume method
    \item Remember staggered grid!
    \item Boundary conditions
    \item Choice of time step for stability
    \item Solving the systems of equations with Eigen
    \item Checking if steady-state has been reached
    \item Define stream function and vorticity.
    \item Transforming results to vorticity streamline formulation
\end{itemize}

\subsection{Bingham plastic fluid}
\label{subsec:bingham}

\begin{itemize}
    \item Transient solution possible, cite appropriate paper and explain
        difficulty in discretizing the viscuous term
    \item Reynolds number zero leads to removal of time-dependency because of
        alternative nondimensionalization of pressure 
    \item Treatment of singularity in effective viscosity: regularization
    \item Finite volume method w/o need for staggered grid, discretization of
        viscuous term
    \item Solution of steady-state system: SIMPLE and its extensions
\end{itemize}

\section{Results}
\label{sec:results}

\subsection{Transient behaviour}
\label{subsec:transient}

\begin{itemize}
    \item Impulsively started
    \item What happens as a function of time? 
    \item Results for different Re? 
\end{itemize}

\subsection{Steady-state solution}
\label{subsec:steady}

All results for Re=100,400,1000,3200\ldots\ldots

\begin{itemize}
    \item 1D slices in the geometric center, including Ghia's results
    \item Stram lines and velocity vector fields
    \item Vorticity 
    \item (Pressure field)!
\end{itemize}

%\begin{figure*}[t]
%    \centering
%    \tikzsetnextfilename{uSlice}
%    \begin{tikzpicture}
%        \begin{groupplot}
%            [
%                height=0.8\columnwidth,
%                width=\columnwidth,
%                group style=
%                {
%                    group size=2 by 2, 
%                    xlabels at=edge bottom, 
%                    xticklabels at=edge bottom, 
%                    ylabels at=edge left, 
%                    vertical sep=25pt, 
%                    horizontal sep=25pt
%                },
%                xlabel=$y$, xmin=0, xmax=1, 
%                ylabel=$u$, 
%                ylabel style={rotate=-90}
%            ]
%            \nextgroupplot[title={Re=100}]
%            \foreach \N in {17,33,65,129}
%            {
%                \addplot+[only marks] table{../Results/uGC_N\N_Re100.out}; 
%            }
%            \addplot+[only marks] table[x={y}, y={Re100}]{../Results/uGC.ref};
%
%            \nextgroupplot[title={Re=400}]
%            \foreach \N in {17,33,65,129}
%            {
%                \addplot+[only marks] table{../Results/uGC_N\N_Re400.out}; 
%            }
%            \addplot+[only marks] table[x={y}, y={Re400}]{../Results/uGC.ref};
%
%            \nextgroupplot[title={Re=1000}]
%            \foreach \N in {17,33,65,129}
%            {
%                \addplot+[only marks] table{../Results/uGC_N\N_Re1000.out}; 
%            }
%            \addplot+[only marks] table[x={y}, y={Re1000}]{../Results/uGC.ref};
%
%            \nextgroupplot[title={Re=3200}]
%            \foreach \N in {17,33,65,129}
%            {
%                \addplot+[only marks] table{../Results/uGC_N\N_Re3200.out}; 
%            }
%            \addplot+[only marks, unbounded coords=jump, 
%            y filter/.expression={y<-0.5 ? nan : y}] 
%            table[x={y}, y={Re3200}]{../Results/uGC.ref};
%        \end{groupplot}
%    \end{tikzpicture}
%    \caption{First component of velocity through $x=0.5$.}
%    \label{fig:uGC}
%\end{figure*}
%
%\begin{figure*}[t]
%    \centering
%    \tikzsetnextfilename{vSlice}
%    \begin{tikzpicture}
%        \begin{groupplot}
%            [
%                height=0.8\columnwidth,
%                width=\columnwidth,
%                group style=
%                {
%                    group size=2 by 2, 
%                    xlabels at=edge bottom, 
%                    xticklabels at=edge bottom, 
%                    ylabels at=edge left, 
%                    vertical sep=25pt, 
%                    horizontal sep=25pt
%                },
%                xlabel=$x$, xmin=0, xmax=1, 
%                ylabel=$v$, 
%                ylabel style={rotate=-90}
%            ]
%            \nextgroupplot[title={Re=100}]
%            \foreach \N in {17,33,65,129}
%            {
%                \addplot+[only marks] table{../Results/vGC_N\N_Re100.out}; 
%            }
%            \addplot+[only marks] table[x={x}, y={Re100}]{../Results/vGC.ref};
%
%            \nextgroupplot[title={Re=400}]
%            \foreach \N in {17,33,65,129}
%            {
%                \addplot+[only marks] table{../Results/vGC_N\N_Re400.out}; 
%            }
%            \addplot+[only marks] table[x={x}, y={Re400}]{../Results/vGC.ref};
%
%            \nextgroupplot[title={Re=1000}]
%            \foreach \N in {17,33,65,129}
%            {
%                \addplot+[only marks] table{../Results/vGC_N\N_Re1000.out}; 
%            }
%            \addplot+[only marks] table[x={x}, y={Re1000}]{../Results/vGC.ref};
%
%            \nextgroupplot[title={Re=3200}]
%            \foreach \N in {17,33,65,129}
%            {
%                \addplot+[only marks] table{../Results/vGC_N\N_Re3200.out}; 
%            }
%            \addplot+[only marks] table[x={x}, y={Re3200}]{../Results/vGC.ref};
%        \end{groupplot}
%    \end{tikzpicture}
%    \caption{Second component of velocity through $y=0.5$.}
%    \label{fig:vGC}
%\end{figure*}
%
%\begin{figure*}[t]
%    \centering
%    \foreach \Re in {100,400,1000,3200}
%    {
%        \pgfplotstableread{../Results/psi_N129_Re\Re.out}{\psi}
%        \tikzsetnextfilename{Stream_Re\Re}
%        \begin{tikzpicture}
%            \begin{axis}
%            [
%                title={${\rm Re}=\Re$},
%                xlabel={$x$}, ylabel={$y$},
%                xmin=0, xmax=1, ymin=0, ymax=1,
%                view={0}{90}
%            ]
%                \addplot3[surf,shader=interp] table[]{\psi}; 
%                \addplot3
%                [
%                    mesh/rows=128, mesh/cols=128, 
%                    contour gnuplot=
%                    {
%                        labels=false, 
%                        draw color=black, 
%                        levels=
%                        {
%                            -1e-10, -1e-7, -1e-5, -1e-4, -.01, -.03, -.05, -.07,
%                            -.09, -.1, -.11, -.115, -.1175, 1e-8, 1e-7, 1e-6, 1e-5,
%                            5e-5, 1e-4, 2.5e-4, 5e-4, 1e-3, 1.5e-3, 3e-3 
%                        }
%                    }
%                ]
%                table{\psi}; 
%                \addplot[white, quiver={u=\thisrow{u},v=\thisrow{v}, scale arrows=0.5}, -stealth]
%                table[each nth point=510]{../Results/vec_N129_Re\Re.out};
%                \addplot[white, quiver={u=\thisrow{u},v=\thisrow{v}, scale arrows=0.5}, -stealth]
%                table[each nth point=514]{../Results/vec_N129_Re\Re.out};
%            \end{axis}
%        \end{tikzpicture}
%    }
%    \caption{Stream function for different Reynold's numbers}
%    \label{fig:stream}
%\end{figure*}
%
%\begin{figure*}[t]
%    \centering
%    \foreach \Re in {100,400,1000,3200}
%    {
%        \pgfplotstableread{../Results/omega_N129_Re\Re.out}{\omega}
%        \tikzsetnextfilename{Vorticity_Re\Re}
%        \begin{tikzpicture}
%            \begin{axis}
%            [
%                title={${\rm Re}=\Re$},
%                xlabel={$x$}, ylabel={$y$},
%                xmin=0, xmax=1, ymin=0, ymax=1,
%                point meta min=-6, point meta max=6,
%                view={0}{90}
%            ]
%                \addplot3[surf,shader=interp] table[]{\omega}; 
%                \addplot3
%                [
%                    mesh/rows=128, mesh/cols=128, 
%                    contour gnuplot=
%                    {
%                        labels=false, 
%                        draw color=black, 
%                        levels={-3,-2,-1,-0.5,0,0.5,1,2,3,4,5}
%                    }
%                ]
%                table[]{\omega}; 
%            \end{axis}
%        \end{tikzpicture}
%    }
%    \caption{Vorticity for different Reynold's numbers}
%    \label{fig:vorticity}
%\end{figure*}

\subsection{Computational efficiency}
\label{subsec:efficiency}

\begin{itemize}
    \item Computational complexity of the linear systems
    \item Runtime (and no of time steps) as a function of Re and $N$
    \item Plots of $\Delta t$ vs. $N$ for different Re 
\end{itemize}

\section{Discussion}
\label{sec:discussion}

\begin{itemize}
    \item Everything works, results exactly as in literature
    \item Transient method is slow for high Re, SIMPLE could be better
    \item Other improvements include Hockney algorithm and multigrid methods
    \item Discuss stability and computational efficiency of
\end{itemize}

\section{Conclusions}
\label{sec:conclusion}


\section*{References}
\bibliographystyle{elsarticle-num}
\bibliography{references.bib}

\end{document}
